\documentclass[12pt]{report}
\usepackage[utf8]{inputenc}
\usepackage[russian]{babel}
%\usepackage[14pt]{extsizes}
\usepackage{listings}
\usepackage{graphicx}
\usepackage{amsmath,amsfonts,amssymb,amsthm,mathtools} 
\usepackage{pgfplots}
\usepackage{filecontents}
\usepackage{float}
\usepackage{comment}
\usepackage{indentfirst}
\usepackage{eucal}
\usepackage{enumitem}
%s\documentclass[openany]{book}
\frenchspacing

\usepackage{indentfirst} % Красная строка

\usetikzlibrary{datavisualization}
\usetikzlibrary{datavisualization.formats.functions}

\usepackage{amsmath}


% Для листинга кода:
\lstset{ %
	language=c,                 % выбор языка для подсветки (здесь это С)
	basicstyle=\small\sffamily, % размер и начертание шрифта для подсветки кода
	numbers=left,               % где поставить нумерацию строк (слева\справа)
	numberstyle=\tiny,           % размер шрифта для номеров строк
	stepnumber=1,                   % размер шага между двумя номерами строк
	numbersep=5pt,                % как далеко отстоят номера строк от подсвечиваемого кода
	showspaces=false,            % показывать или нет пробелы специальными отступами
	showstringspaces=false,      % показывать или нет пробелы в строках
	showtabs=false,             % показывать или нет табуляцию в строках
	frame=single,              % рисовать рамку вокруг кода
	tabsize=2,                 % размер табуляции по умолчанию равен 2 пробелам
	captionpos=t,              % позиция заголовка вверху [t] или внизу [b] 
	breaklines=true,           % автоматически переносить строки (да\нет)
	breakatwhitespace=false, % переносить строки только если есть пробел
	escapeinside={\#*}{*)}   % если нужно добавить комментарии в коде
}


\usepackage[left=2cm,right=2cm, top=2cm,bottom=2cm,bindingoffset=0cm]{geometry}
% Для измененных титулов глав:
\usepackage{titlesec, blindtext, color} % подключаем нужные пакеты
\definecolor{gray75}{gray}{0.75} % определяем цвет
\newcommand{\hsp}{\hspace{20pt}} % длина линии в 20pt
% titleformat определяет стиль
\titleformat{\chapter}[hang]{\Huge\bfseries}{\thechapter\hsp\textcolor{gray75}{|}\hsp}{0pt}{\Huge\bfseries}


% plot
\usepackage{pgfplots}
\usepackage{filecontents}
\usetikzlibrary{datavisualization}
\usetikzlibrary{datavisualization.formats.functions}

\begin{document}
	%\def\chaptername{} % убирает "Глава"
	\thispagestyle{empty}
	\begin{titlepage}
		\noindent \begin{minipage}{0.15\textwidth}
			\includegraphics[width=\linewidth]{img/b_logo}
		\end{minipage}
		\noindent\begin{minipage}{0.9\textwidth}\centering
			\textbf{Министерство науки и высшего образования Российской Федерации}\\
			\textbf{Федеральное государственное бюджетное образовательное учреждение высшего образования}\\
			\textbf{~~~«Московский государственный технический университет имени Н.Э.~Баумана}\\
			\textbf{(национальный исследовательский университет)»}\\
			\textbf{(МГТУ им. Н.Э.~Баумана)}
		\end{minipage}
		
		\noindent\rule{18cm}{3pt}
		\newline\newline
		\noindent ФАКУЛЬТЕТ $\underline{\text{«Информатика и системы управления»}}$ \newline\newline
		\noindent КАФЕДРА $\underline{\text{«Программное обеспечение ЭВМ и информационные технологии»}}$\newline\newline\newline\newline\newline
		
		\begin{center}
			\noindent\begin{minipage}{1.1\textwidth}\centering
				\Large\textbf{  Отчет по лабораторной работе №13}\newline
				\textbf{по дисциплине <<Функциональное и логическое}\newline
				\textbf{~~~программирование>>}\newline\newline
			\end{minipage}
		\end{center}
		
		\noindent\textbf{Тема} $\underline{\text{Структура программы на Prolog и ее реализация~~~~~~~~~~~~~~}}$\newline\newline
		\noindent\textbf{Студент} $\underline{\text{Криков А.В.~~~~~~~~~~~~~~~~~~~~~~~~~~~~~~~~~~~~~~~}}$\newline\newline
		\noindent\textbf{Группа} $\underline{\text{ИУ7-63Б~~~~~~~~~~~~~~~~~~~~~~~~~~~~~~~~~~~~~~~~~~~~~~~}}$\newline\newline
		\noindent\textbf{Оценка (баллы)} $\underline{\text{~~~~~~~~~~~~~~~~~~~~~~~~~~~~~~~~~~~~~~~~~~~~~~}}$\newline\newline
		\noindent\textbf{Преподаватель} $\underline{\text{Толпинская Н.Б., Строганов Ю. В.}}$\newline\newline\newline
		
		\begin{center}
			\vfill
			Москва~---~\the\year
			~г.
		\end{center}
	\end{titlepage}
	

\chapter*{Лабораторная работа №13}
\section*{Постановка задачи}
Создать базу знаний «Собственники», дополнив базу знаний, хранящую знания (лаб. 13):

\begin{itemize}
	\item <<Телефонный справочник>>: Фамилия, №тел, Адрес – структура (Город, Улица, №дома, №кв),
	\item <<Автомобили>>: Фамилия\_владельца, Марка, Цвет, Стоимость, и др.,
	\item <<Вкладчики банков>>: Фамилия, Банк, счет, сумма, др.,
\end{itemize}

знаниями о дополнительной собственности владельца. Преобразовать знания об автомобиле к форме знаний о собственности.

Вид собственности (кроме автомобиля):

\begin{itemize}
	\item Строение, стоимость и другие его характеристики.
	\item Участок, стоимость и другие его характеристики.
	\item Водный\_транспорт, стоимость и другие его характеристики.
\end{itemize}

Описать и использовать вариантный домен: \textbf{Собственность}. Владелец может иметь, но только один объект каждого вида собственности (это касается и автомобиля), или не иметь некоторых видов собственности. 

Используя конъюнктивное правило и разные формы задания одного вопроса (пояснять для какого №задания – какой вопрос), обеспечить возможность поиска:

\begin{enumerate}
	\item Названий всех объектов собственности заданного субъекта.
	\item Названий и стоимости всех объектов собственности заданного субъекта.
	\item Разработать правило, позволяющее найти суммарную стоимость всех объектов собственности заданного субъекта.
\end{enumerate}

Для 2-го пункта и одной фамилии составить таблицу, отражающую конкретный порядок работы системы, с объяснениями порядка работы и особенностей использования доменов.

\subsection*{Решение}
\begin{lstlisting}
domains
	city, street, phone, surname, name = string
	house, flat = integer
	address = addr(city, street, house, flat)
	mark, color, bank = string
	id, amount, price = integer

	object = building(name, price);
		region(name, price);
		water_transport(mark, color, price);
		car(mark, color, price).

predicates
	phone(surname, phone, address)
	bank_depositor(surname, bank, id, amount)
	owner(surname, object)

	all_property(surname, name)
	all_property_price(surname, name, price)

clauses
	man("Danil", "+798523415232", addr("Moscow", "Bassmannaya", 34, 12)).
	man("Anton", "+79752345123", addr("Moscow", "Lesnaya", 41, 37)).
	man("Leonardo", "+75012354433", addr("Kiev", "Lenina", 73, 13)).
	man("Dmirty", "+79752341432", addr("Sochi", "Sovetskata", 14, 88)).

	owner("Leonardo", car("Lada", "Green", 1000)).
	owner("Leonardo", region("Nothung", 0)).
	owner("Leonardo", building("Moscow-city", 100500)).
	owner("Anton", car("Lada", "green", 1000)).
	owner("Anton", region("Kiev", 10000)).
	owner("Anton", building("Mail.ru Office", 20000)).
	owner("Anton", water_transport("Yacht", "Red", 10000)).
	owner("Dmitry", car("Cadillac", "Black", 304000)).
	owner("Dmitry", building("BMSTU", 200000)).
	owner("Danil", car("Mercedes", "White", 30000)).
	owner("Danil", building("Tree", 10)).
	
	deposit("Leonardo", "Tinkoff", 22, 1000).
	deposit("Dmitry", "Sber", 33, 10000).
	deposit("Dmitry", "Alfa", 44, 20000).
	deposit("Anton", "VTB", 238, 10).
	deposit("Danil", "PochtaBank", 1, 10000).
	
	all_property(Surname, Name) :- owner(Surname, car(Name, _, _)).
	all_property(Surname, Name) :- owner(Surname, building(Name, _)).
	all_property(Surname, Name) :- owner(Surname, region(Name, _)).
	all_property(Surname, Name) :- owner(Surname, water_transport(Name, _, _)).
	
	all_property_price(Surname, Name, Price) :- owner(Surname, car(Name, _, Price)).
	all_property_price(Surname, Name, Price) :- owner(Surname, building(Name, Price)).
	all_property_price(Surname, Name, Price) :- owner(Surname, region(Name, Price)).
	all_property_price(Surname, Name, Price) :- owner(Surname, water_transport(Name, _, Price)).

goal
	%all_property("Anton", Name).
	%all_property_with_price("Danil", Name, Price).

\end{lstlisting}


\bibliographystyle{utf8gost705u}  % стилевой файл для оформления по ГОСТу
\bibliography{51-biblio}          % имя библиографической базы (bib-файла)
	
\end{document}
