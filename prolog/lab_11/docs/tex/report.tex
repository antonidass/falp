\documentclass[12pt]{report}
\usepackage[utf8]{inputenc}
\usepackage[russian]{babel}
%\usepackage[14pt]{extsizes}
\usepackage{listings}
\usepackage{graphicx}
\usepackage{amsmath,amsfonts,amssymb,amsthm,mathtools} 
\usepackage{pgfplots}
\usepackage{filecontents}
\usepackage{float}
\usepackage{comment}
\usepackage{indentfirst}
\usepackage{eucal}
\usepackage{enumitem}
%s\documentclass[openany]{book}
\frenchspacing

\usepackage{indentfirst} % Красная строка

\usetikzlibrary{datavisualization}
\usetikzlibrary{datavisualization.formats.functions}

\usepackage{amsmath}


% Для листинга кода:
\lstset{ %
	language=c,                 % выбор языка для подсветки (здесь это С)
	basicstyle=\small\sffamily, % размер и начертание шрифта для подсветки кода
	numbers=left,               % где поставить нумерацию строк (слева\справа)
	numberstyle=\tiny,           % размер шрифта для номеров строк
	stepnumber=1,                   % размер шага между двумя номерами строк
	numbersep=5pt,                % как далеко отстоят номера строк от подсвечиваемого кода
	showspaces=false,            % показывать или нет пробелы специальными отступами
	showstringspaces=false,      % показывать или нет пробелы в строках
	showtabs=false,             % показывать или нет табуляцию в строках
	frame=single,              % рисовать рамку вокруг кода
	tabsize=2,                 % размер табуляции по умолчанию равен 2 пробелам
	captionpos=t,              % позиция заголовка вверху [t] или внизу [b] 
	breaklines=true,           % автоматически переносить строки (да\нет)
	breakatwhitespace=false, % переносить строки только если есть пробел
	escapeinside={\#*}{*)}   % если нужно добавить комментарии в коде
}


\usepackage[left=2cm,right=2cm, top=2cm,bottom=2cm,bindingoffset=0cm]{geometry}
% Для измененных титулов глав:
\usepackage{titlesec, blindtext, color} % подключаем нужные пакеты
\definecolor{gray75}{gray}{0.75} % определяем цвет
\newcommand{\hsp}{\hspace{20pt}} % длина линии в 20pt
% titleformat определяет стиль
\titleformat{\chapter}[hang]{\Huge\bfseries}{\thechapter\hsp\textcolor{gray75}{|}\hsp}{0pt}{\Huge\bfseries}


% plot
\usepackage{pgfplots}
\usepackage{filecontents}
\usetikzlibrary{datavisualization}
\usetikzlibrary{datavisualization.formats.functions}

\begin{document}
	\thispagestyle{empty}
	\begin{titlepage}
		\noindent \begin{minipage}{0.15\textwidth}
			\includegraphics[width=\linewidth]{img/b_logo}
		\end{minipage}
		\noindent\begin{minipage}{0.9\textwidth}\centering
			\textbf{Министерство науки и высшего образования Российской Федерации}\\
			\textbf{Федеральное государственное бюджетное образовательное учреждение высшего образования}\\
			\textbf{~~~«Московский государственный технический университет имени Н.Э.~Баумана}\\
			\textbf{(национальный исследовательский университет)»}\\
			\textbf{(МГТУ им. Н.Э.~Баумана)}
		\end{minipage}
		
		\noindent\rule{18cm}{3pt}
		\newline\newline
		\noindent ФАКУЛЬТЕТ $\underline{\text{«Информатика и системы управления»}}$ \newline\newline
		\noindent КАФЕДРА $\underline{\text{«Программное обеспечение ЭВМ и информационные технологии»}}$\newline\newline\newline\newline\newline
		
		\begin{center}
			\noindent\begin{minipage}{1.1\textwidth}\centering
				\Large\textbf{  Отчет по лабораторной работе №11}\newline
				\textbf{по дисциплине <<Функциональное и логическое}\newline
				\textbf{~~~программирование>>}\newline\newline
			\end{minipage}
		\end{center}
		
		\noindent\textbf{Тема} $\underline{\text{Среда Visual Prolog. Структура программы. Работа программы}}$\newline\newline
		\noindent\textbf{Студент} $\underline{\text{Криков А.В.~~~~~~~~~~~~~~~~~~~~~~~~~~~~~~~~~~~~~~~~~~~~~~~~~~~~~~~~~~~~~~~~~}}$\newline\newline
		\noindent\textbf{Группа} $\underline{\text{ИУ7-63Б~~~~~~~~~~~~~~~~~~~~~~~~~~~~~~~~~~~~~~~~~~~~~~~~~~~~~~~~~~~~~~~~~~~~~~~~~}}$\newline\newline
		\noindent\textbf{Оценка (баллы)} $\underline{\text{~~~~~~~~~~~~~~~~~~~~~~~~~~~~~~~~~~~~~~~~~~~~~~~~~~~~~~~~~~~~~~~~~~~~~~~~}}$\newline\newline
		\noindent\textbf{Преподаватель} $\underline{\text{Толпинская Н.Б., Строганов Ю. В.~~~~~~~~~~~~~~~~~~~~~~~~~~}}$\newline\newline\newline
		
		\begin{center}
			\vfill
			Москва~---~\the\year
			~г.
		\end{center}
	\end{titlepage}
	




	\section*{Задание 1}
	Разработать свою программу — <<Телефонный справочник>>. Протестировать работы программы.
	

\section*{Задание 2}
Составить программу — базу знаний, с помощью которой можно определить, например, множество студентов, обучающихся в одном ВУЗе. Студент может одновременно обучаться в нескольких ВУЗах. Привести примеры возможных вариантов вопросов и варианты ответов (не менее 3-х), Описать порядок формирования вариантов ответа.

\subsection*{Решение}

Данная база знаний содержит информацию о студентах (фамилия, вуз, успеваемость, группа).\\

\subsubsection*{Запрос 1}
Студенты, которые учатся в МАИ и их успеваемость. Происходит проход сверху вниз по всем фактам предиката \emph{student(surname, university, grade, group)} и осуществляется унификация с \emph{student(Surname, "MAI"{}, Grade, \_)}. Унификацию успешно проходит один факт: \emph{student("Ivanov"{}, "MAI"{}, "D"{}, "MA-11"{})}.\\

\subsubsection*{Запрос 2}
Все неуспевающие студенты МГТУ. Происходит проход по всем фактам предиката \emph{student(surname, university, grade, group)} и осуществляется унификация с \emph{student(Surname, "BMSTU"{}, "D"{}, \_)}.  Успешно унификацию проходят факты \emph{student("Alexandrov"{}, "BMSTU"{}, "D"{}, "IU6-61B"{})} и \emph{student("Filipov"{}, "BMSTU"{}, "D"{}, "MT-14B"{})}.\\

\subsubsection*{Запрос 3}
Все студенты всех ВУЗ-ов.  Происходит проход по всем фактам предиката \emph{student(surname, university, grade, group)} и осуществляется унификация с \emph{student(Surname, University, Grade, Group)}.  Успешно унификацию проходят все записи.\\


\chapter*{Теоретическая часть}

\section*{1. Что собой представляет программа на языке пролог?}

Программа на Prolog представляет собой набор фактов и правил, обеспечивающих получение заключений на основе этих утверждений. Программа содержит базу знаний и вопрос. База знаний содержит истинные значения, используя которые программа выдает ответ на вопрос. 

Основным элементом языка является терм. База знаний состоит из предложений. Каждое предложение заканчивается точкой. Вопрос состоит только из тела – составного терма (или нескольких составных термов). Вопросы используются для выяснения выполнимости некоторого отношения между описанными в программе объектами. Система рассматривает вопрос как цель, к которой (к истинности которой) надо стремиться. Ответ на вопрос может оказаться логически положительным или отрицательным, в зависимости от того, может ли быть достигнута соответствующая цель.

\section*{2. Какова структура программы на Prolog?}

Программа на Prolog состоит из следующих разделов:

\begin{itemize}
	\item директивы компилятора — зарезервированные символьные константы,
	\item CONSTANTS — раздел описания констант,
	\item DOMAINS — раздел описания доменов,
	\item DATABASE — раздел описания предикатов внутренней базы данных,
	\item PREDICATES — раздел описания предикатов,
	\item CLAUSES — раздел описания предложений базы знаний,
	\item GOAL — раздел описания внутренней цели (вопроса).
	В программе не обязательно должны быть все разделы.
\end{itemize}

\section*{3. Как реализуется программа на Prolog? Как формируются результаты работы программы?}

Ответ на поставленный вопрос система дает в логической форме - «Да» или «Нет». Цель системы состоит в том, чтобы на поставленный вопрос найти возможность, исходя из базы знаний, ответить «Да». Вариантов ответить «Да» на поставленный вопрос может быть несколько. В нашем случае система настроена в режим получения всех возможных вариантов ответа. При поиске ответов на вопрос рассматриваются альтернативные варианты и находятся все возможные решения (методом проб и ошибок) - множества значений переменных, при которых на поставленный вопрос можно ответить - «Да».

Для выполнения логического вывода используется механизм унификации, встроенный в систему.
Унификация – операция, которая позволяет формализовать процесс логического вывода. С практической точки зрения  - это основной вычислительный шаг, с помощью которого происходит:
\begin{itemize}
	\item Двунаправленная передача параметров процедурам,
	\item Неразрушающее присваивание,
	\item Проверка условий (доказательство).
\end{itemize}

В процессе работы система выполняет большое число унификаций.  Попытка "увидеть одинаковость" – сопоставимость двух термов, может завершаться успехом или тупиковой ситуацией (неудачей). В последнем случае включается механизм отката к предыдущему шагу.



\newpage
\subsection*{Исходный код к лабораторной №11(1)}
\begin{lstlisting}
	/* LAB 11(1) */

domains 
	 name, phone = symbol.
	 age = integer.
	
predicates 
	 record(name, age, phone)
	
clauses
	 record("Anton", 21, "+79061212200").
	 record("Danil", 30, "+79257294819").
	 record("Maria", 18, "+79152385934").
	 record("Alex", 74, "+79275928501").
	 record("Oleg", 33, "+79523057392").
	 record("Petr", 99, "+79210572022").
	 record("Anton", 99, "+79210572022").
	 
goal 
	 %record("Anton", 21, "+79061212200").
	 record("Anton", Age, _).
   
\end{lstlisting}


\subsection*{Исходный код к лабораторной №11(2)}
\begin{lstlisting}
	/* LAB 11(2) */
domains 
	 surname, university, grade, group = symbol.
	 
predicates
	 student(surname, university, grade, group)
	 
clauses
	 student("Krikov", "BMSTU", "B", "IU7-63B").
	 student("Petrov", "BMSTU", "A", "IU7-62B").
	 student("Ivanov", "MAI", "D", "MA-11").
	 student("Ivanov", "BMSTU", "A", "IU7-41B").
	 student("Sidorov", "BMSTU", "A", "IU7-63B").
	 student("Popov", "MSU", "B", "MM-45").
	 student("Alexandrov", "BMSTU", "D", "IU6-61B").
	 student("Dubov", "MSU", "C", "VM-23").   
	 student("Filipov", "BMSTU", "D", "MT-14B").
	 
goal
	 student(Surname, "MAI", Grade, _).
	 %student(Surname, "BMSTU", "D", _).
	 %student(Surname, University, Grade, Group).
   
\end{lstlisting}



\bibliographystyle{utf8gost705u}  % стилевой файл для оформления по ГОСТу
\bibliography{51-biblio}          % имя библиографической базы (bib-файла)
	
\end{document}
