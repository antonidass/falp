\documentclass[12pt]{report}
\usepackage[utf8]{inputenc}
\usepackage[russian]{babel}
%\usepackage[14pt]{extsizes}
\usepackage{listings}
\usepackage{graphicx}
\usepackage{amsmath,amsfonts,amssymb,amsthm,mathtools} 
\usepackage{pgfplots}
\usepackage{filecontents}
\usepackage{float}
\usepackage{indentfirst}
\usepackage{eucal}
\usepackage{enumitem}
\usepackage{tasks}

%s\documentclass[openany]{book}
\frenchspacing

\usepackage{indentfirst} % Красная строка

\usetikzlibrary{datavisualization}
\usetikzlibrary{datavisualization.formats.functions}

\usepackage{amsmath}


% Для листинга кода:
\lstset{ %
	language=c,                 % выбор языка для подсветки (здесь это С)
	basicstyle=\small\sffamily, % размер и начертание шрифта для подсветки кода
	numbers=left,               % где поставить нумерацию строк (слева\справа)
	numberstyle=\tiny,           % размер шрифта для номеров строк
	stepnumber=1,                   % размер шага между двумя номерами строк
	numbersep=5pt,                % как далеко отстоят номера строк от подсвечиваемого кода
	showspaces=false,            % показывать или нет пробелы специальными отступами
	showstringspaces=false,      % показывать или нет пробелы в строках
	showtabs=false,             % показывать или нет табуляцию в строках
	frame=single,              % рисовать рамку вокруг кода
	tabsize=2,                 % размер табуляции по умолчанию равен 2 пробелам
	captionpos=t,              % позиция заголовка вверху [t] или внизу [b] 
	breaklines=true,           % автоматически переносить строки (да\нет)
	breakatwhitespace=false, % переносить строки только если есть пробел
	escapeinside={\#*}{*)}   % если нужно добавить комментарии в коде
}


\usepackage[left=2cm,right=2cm, top=2cm,bottom=2cm,bindingoffset=0cm]{geometry}
% Для измененных титулов глав:
\usepackage{titlesec, blindtext, color} % подключаем нужные пакеты
\definecolor{gray75}{gray}{0.75} % определяем цвет
\newcommand{\hsp}{\hspace{20pt}} % длина линии в 20pt
% titleformat определяет стиль
\titleformat{\chapter}[hang]{\Huge\bfseries}{\thechapter\hsp\textcolor{gray75}{|}\hsp}{0pt}{\Huge\bfseries}


% plot
\usepackage{pgfplots}
\usepackage{filecontents}
\usetikzlibrary{datavisualization}
\usetikzlibrary{datavisualization.formats.functions}

\begin{document}
	%\def\chaptername{} % убирает "Глава"
	\thispagestyle{empty}
	\begin{titlepage}
		\noindent \begin{minipage}{0.15\textwidth}
			\includegraphics[width=\linewidth]{img/b_logo}
		\end{minipage}
		\noindent\begin{minipage}{0.9\textwidth}\centering
			\textbf{Министерство науки и высшего образования Российской Федерации}\\
			\textbf{Федеральное государственное бюджетное образовательное учреждение высшего образования}\\
			\textbf{~~~«Московский государственный технический университет имени Н.Э.~Баумана}\\
			\textbf{(национальный исследовательский университет)»}\\
			\textbf{(МГТУ им. Н.Э.~Баумана)}
		\end{minipage}
		
		\noindent\rule{18cm}{3pt}
		\newline\newline
		\noindent ФАКУЛЬТЕТ $\underline{\text{«Информатика и системы управления»}}$ \newline\newline
		\noindent КАФЕДРА $\underline{\text{«Программное обеспечение ЭВМ и информационные технологии»}}$\newline\newline\newline\newline\newline
		
		\begin{center}
			\noindent\begin{minipage}{1.1\textwidth}\centering
				\Large\textbf{  Отчет по лабораторной работе №6}\newline
				\textbf{по дисциплине <<Функциональное и логическое}\newline
				\textbf{~~~программирование>>}\newline\newline
			\end{minipage}
		\end{center}
		
		\noindent\textbf{Тема} $\underline{\text{Использование функционалов}}$\newline\newline
		\noindent\textbf{Студент} $\underline{\text{Криков А.В.~~~~~~~~~~~~~~~~~~~~~~~~~~~~~~~~~~~~~~~~~~}}$\newline\newline
		\noindent\textbf{Группа} $\underline{\text{ИУ7-63Б~~~~~~~~~~~~~~~~~~~~~~~~~~~~~~~~~~~~~~~~~~~~~~~~~~}}$\newline\newline
		\noindent\textbf{Оценка (баллы)} $\underline{\text{~~~~~~~~~~~~~~~~~~~~~~~~~~~~~~~~~~~~~~~~~~~~~~~~~}}$\newline\newline
		\noindent\textbf{Преподаватель} $\underline{\text{Толпинская Н.Б., Строганов Ю.В.~~~~~~~~~~~~~~~~~~~~~~~~~~~~}}$\newline\newline\newline
		
		\begin{center}
			\vfill
			Москва~---~\the\year
			~г.
		\end{center}
	\end{titlepage}
	
	


\chapter*{Практические задания}

	
\section*{Задание 1}
Напишите функцию, которая уменьшает на 10 все числа из списка-аргумента этой функции

\begin{lstlisting}[language=Lisp]
(defun minus-10 (lst)
	(mapcar (lambda (e) (if (numberp e) (- e 10) e)) lst))  
\end{lstlisting}


\section*{Задание 2}
Напишите функцию, которая умножает на заданное число-аргумент все числа из заданного списка-аргумента, когда

\begin{enumerate}
	\item Все элементы списка --- числа,
	\item Элементы списка --- любые объекты.
\end{enumerate}

\begin{lstlisting}[language=Lisp]
(defun mult-all (lst n)
	(mapcar (lambda (e) (if (numberp e) (* e n) e)) lst))
  
	
(defun mult-all-deep (lst n)
	(mapcar (lambda (e) (cond ((numberp e) (* e n))
				  ((listp e) (mult-all-deep e n))
				  (T e)))
		lst))
			  
\end{lstlisting}



\section*{Задание 3}
Написать функцию, которая по своему списку-аргументу lst определяет является ли он палиндромом (то есть равны ли lst и (reverse lst)).
\begin{lstlisting}[language=Lisp]
(defun my-reverse (lst)
	(reduce (lambda (res e) (cons e res)) lst :initial-value ()))
	
(defun palindromp (lst)
	(equal lst (my-reverse lst)))
	
\end{lstlisting}


\section*{Задание 4}
Написать предикат set-equal, который возвращает t, если два его множествааргумента содержат одни и те же элементы, порядок которых не имеет значения.

\begin{lstlisting}[language=Lisp]
(defun in-set (el lst)
	(reduce (lambda (a b) (or a b))
		(mapcar (lambda (e) (equal el e)) lst)))
  
(defun my-subsetp (l1 l2)
	(reduce (lambda (a b) (and a b))
		(mapcar (lambda (e) (in-set e l2)) l1)))
  
(defun set-equal (l1 l2)
	(and (my-subsetp l1 l2) (my-subsetp l2 l1)))  
\end{lstlisting}



\section*{Задание 5}
Написать функцию которая получает как аргумент список чисел, а возвращает список квадратов этих чисел в том же порядке.

\begin{lstlisting}[language=Lisp]
(defun sqr-lst (lst)
	(mapcar (lambda (e) (* e e)) lst)) 
\end{lstlisting}


\section*{Задание 6}
Напишите функцию, \texttt{select-between}, которая из списка-аргумента, содержащего только числа, выбирает только те, которые расположены между двумя указанными границамиаргументами и возвращает их в виде списка (упорядоченного по возрастанию списка чисел (+ 2 балла)).

\begin{lstlisting}[language=Lisp]
(defun select-between (lst n1 n2)
	(mapcan (lambda (e) (and (< n1 e n2) (list e)))
		lst))
  
(defun select-between-sorted (lst n1 n2)
	(sort (select-between lst n1 n2) #'<))

\end{lstlisting}






\section*{Задание 7}
Написать функцию, вычисляющую декартово произведение двух своих списковаргументов.


\begin{lstlisting}[language=Lisp]
(defun decart-mult (l1 l2)
	(mapcan (lambda (e1)
		  (mapcar (lambda (e2)
				(list e1 e2))
			  l2))
		l1))
\end{lstlisting}



\section*{Задание 8}
Почему так реализовано \texttt{reduce}, в чем причина? \texttt{(reduce \#'+ ()) -> 0}

\subsection*{Решение}
Функция \texttt{+} --- функционал, который при 0 количестве аргументов возвращает значение 0. Если подать на вход \texttt{reduce} функцию, которая не может обработать 0 аргументов, то вызов \texttt{reduce} с пустым списком в качестве второго аргумента вернет ошибку (\texttt{invalid number of arguments: 0}). При этом, если подано более одного аргумента, то \texttt{reduce} выполняет действия:

\begin{enumerate}
	\item Сохраняет первый элемент списка в область памяти;
	\item Для всех остальных элементов списка выполняет переданную в качестве первого аргумента функцию, подавая на вход 2 аргумента и сохраняя результат в \texttt{acc}.
\end{enumerate}




\section*{Задание 9}
Пусть \texttt{list-of-list} список, состоящий из списков. Написать функцию, которая вычисляет сумму длин всех элементов \texttt{list-of-list}, т.е. например для аргумента ((1 2) (3 4)) -> 4.

\begin{lstlisting}[language=Lisp]
(defun sum-of-len (lst)
  (reduce (lambda (res e) (+ res (length e)))
	  lst
	  :initial-value 0))
\end{lstlisting}


\bibliographystyle{utf8gost705u}  % стилевой файл для оформления по ГОСТу
	
\bibliography{51-biblio}          % имя библиографической базы (bib-файла)
	
\end{document}