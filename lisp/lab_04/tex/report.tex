\documentclass[12pt]{report}
\usepackage[utf8]{inputenc}
\usepackage[russian]{babel}
%\usepackage[14pt]{extsizes}
\usepackage{listings}
\usepackage{graphicx}
\usepackage{amsmath,amsfonts,amssymb,amsthm,mathtools} 
\usepackage{pgfplots}
\usepackage{filecontents}
\usepackage{float}
\usepackage{indentfirst}
\usepackage{eucal}
\usepackage{enumitem}
\usepackage{tasks}

%s\documentclass[openany]{book}
\frenchspacing

\usepackage{indentfirst} % Красная строка

\usetikzlibrary{datavisualization}
\usetikzlibrary{datavisualization.formats.functions}

\usepackage{amsmath}


% Для листинга кода:
\lstset{ %
	language=c,                 % выбор языка для подсветки (здесь это С)
	basicstyle=\small\sffamily, % размер и начертание шрифта для подсветки кода
	numbers=left,               % где поставить нумерацию строк (слева\справа)
	numberstyle=\tiny,           % размер шрифта для номеров строк
	stepnumber=1,                   % размер шага между двумя номерами строк
	numbersep=5pt,                % как далеко отстоят номера строк от подсвечиваемого кода
	showspaces=false,            % показывать или нет пробелы специальными отступами
	showstringspaces=false,      % показывать или нет пробелы в строках
	showtabs=false,             % показывать или нет табуляцию в строках
	frame=single,              % рисовать рамку вокруг кода
	tabsize=2,                 % размер табуляции по умолчанию равен 2 пробелам
	captionpos=t,              % позиция заголовка вверху [t] или внизу [b] 
	breaklines=true,           % автоматически переносить строки (да\нет)
	breakatwhitespace=false, % переносить строки только если есть пробел
	escapeinside={\#*}{*)}   % если нужно добавить комментарии в коде
}


\usepackage[left=2cm,right=2cm, top=2cm,bottom=2cm,bindingoffset=0cm]{geometry}
% Для измененных титулов глав:
\usepackage{titlesec, blindtext, color} % подключаем нужные пакеты
\definecolor{gray75}{gray}{0.75} % определяем цвет
\newcommand{\hsp}{\hspace{20pt}} % длина линии в 20pt
% titleformat определяет стиль
\titleformat{\chapter}[hang]{\Huge\bfseries}{\thechapter\hsp\textcolor{gray75}{|}\hsp}{0pt}{\Huge\bfseries}


% plot
\usepackage{pgfplots}
\usepackage{filecontents}
\usetikzlibrary{datavisualization}
\usetikzlibrary{datavisualization.formats.functions}

\begin{document}
	%\def\chaptername{} % убирает "Глава"
	\thispagestyle{empty}
	\begin{titlepage}
		\noindent \begin{minipage}{0.15\textwidth}
			\includegraphics[width=\linewidth]{img/b_logo}
		\end{minipage}
		\noindent\begin{minipage}{0.9\textwidth}\centering
			\textbf{Министерство науки и высшего образования Российской Федерации}\\
			\textbf{Федеральное государственное бюджетное образовательное учреждение высшего образования}\\
			\textbf{~~~«Московский государственный технический университет имени Н.Э.~Баумана}\\
			\textbf{(национальный исследовательский университет)»}\\
			\textbf{(МГТУ им. Н.Э.~Баумана)}
		\end{minipage}
		
		\noindent\rule{18cm}{3pt}
		\newline\newline
		\noindent ФАКУЛЬТЕТ $\underline{\text{«Информатика и системы управления»}}$ \newline\newline
		\noindent КАФЕДРА $\underline{\text{«Программное обеспечение ЭВМ и информационные технологии»}}$\newline\newline\newline\newline\newline
		
		\begin{center}
			\noindent\begin{minipage}{1.1\textwidth}\centering
				\Large\textbf{  Отчет по лабораторной работе №4}\newline
				\textbf{по дисциплине <<Функциональное и логическое}\newline
				\textbf{~~~программирование>>}\newline\newline
			\end{minipage}
		\end{center}
		
		\noindent\textbf{Тема} $\underline{\text{Использование управляющих структур, работа со списками}}$\newline\newline
		\noindent\textbf{Студент} $\underline{\text{Криков А.В.~~~~~~~~~~~~~~~~~~~~~~~~~~~~~~~~~~~~~~~~~~}}$\newline\newline
		\noindent\textbf{Группа} $\underline{\text{ИУ7-63Б~~~~~~~~~~~~~~~~~~~~~~~~~~~~~~~~~~~~~~~~~~~~~~~~~~}}$\newline\newline
		\noindent\textbf{Оценка (баллы)} $\underline{\text{~~~~~~~~~~~~~~~~~~~~~~~~~~~~~~~~~~~~~~~~~~~~~~~~~}}$\newline\newline
		\noindent\textbf{Преподаватель} $\underline{\text{Толпинская Н.Б., Строганов Ю.В.~~~~~~~~~~~~~~~~~~~~~~~~~~~~}}$\newline\newline\newline
		
		\begin{center}
			\vfill
			Москва~---~\the\year
			~г.
		\end{center}
	\end{titlepage}
	
	


\chapter*{Практические задания}

\section*{Задание 1}
Каковы результаты вычисления следующих выражений?

\begin{lstlisting}[label=9, language=lisp]
(setf lst1 '(a b))
(setf lst2 '(c d))

(cons lst1 lst2)   ; (a b) . (c d) ->  ((a b) c d)
(list lst1 lst2)   ; ((a b) (c d))
(append lst1 lst2) ; (a b c d)

\end{lstlisting}
	
	

\section*{Задание 2}
Каковы результаты вычисления следующих выражений, и почему?

\begin{lstlisting}[label=9, language=lisp]
(reverse ())         ; NIL
(last ())            ; NIL
(reverse '(a))       ; (A)
(last '(a))          ; (A)
(reverse '((a b c))) ; ((A B C))
\end{lstlisting}

\section*{Задание 3}
Написать, по крайней мере, два варианта функции, которая возвращает последний элемент
своего списка-аргумента.

\begin{lstlisting}[label=9, language=lisp]
(defun last1 (a) (first (last a)))
(defun last2 (a) (first (reverse a)))
\end{lstlisting}


\clearpage

\section*{Задание 4}
Написать, по крайней мере, два варианта функции, которая возвращает свой список-аргумент без последнего элемента.

\begin{lstlisting}[label=9, language=lisp]
(defun delete_last (a) 
(reverse (cdr (reverse a))))

(defun delete_last_2 (a)
(butlast a))

(defun delete_last_3(a)                      
  (cond 
    ((null (cdr a)) nil)                        
    (t (cons (car a) (reduce_list_3 (cdr a)))
    ))
)
\end{lstlisting}



\section*{Задание 5}
Написать простой вариант игры в кости, в котором бросаются две правильные кости. Если
сумма выпавших очков равна 7 или 11 -- выигрыш, если выпало (1,1) или (6,6) --- игрок право
снова бросить кости, во всех остальных случаях ход переходит ко второму игроку, но
запоминается сумма выпавших очков. Если второй игрок не выигрывает абсолютно, то
выигрывает тот игрок, у которого больше очков. Результат игры и значения выпавших костей
выводить на экран с помощью функции print.

\begin{lstlisting}[label=9, language=lisp]
(defun random_score ()
(list (+ (random 5) 1) (+ (random 5) 1)))

(defun check_sum_to_replay(result)
(if (or (equal result '(6 6)) (equal result '(1 1))) T NIL))

(defun print_result_to_replay (result)
(print "Score: ") 
(prin1 result) 
(print "The dice will be rerolled...") 
(print "-------"))

(defun make_player_score ()
(let* ((result (random_score)))
    (if (check_sum_to_replay result) 
        (and (print_result_to_replay result) 
            (make_player_score)) result)))

(defun sum (result)
(+ (first result) (second result)))

(defun check_sum_to_win (result)
(if (or (equal (sum result) 7) 
(equal (sum result) 11)) T NIL))

(defun print_player (result)
(print "Score: ")
(PRIN1 result))

(defun play()
(print "---First player---")
(let* ((first_val (make_player_score)))
    (if (check_sum_to_win first_val) (and (print_player first_val) "First player won!") 
        (and (print_player first_val)
            (print "------------------")
            (print "---Second player---")
             (let* ((second_val (make_player_score)))
                (if (check_sum_to_win second_val) (and (print_player second_val) "Second player won!")
                    (and (print_player second_val) 
                        (if (>= (sum first_val) (sum second_val)) 
                            "First player won!" "Second player won!")
                        )))))))
					\end{lstlisting}



\section*{Контрольные вопросы}

\subsection*{1. Синтаксическая форма и хранение программы в памяти}

В \texttt{LISP} формы представления программы и обрабатываемых ею данных одинаковы и представляются в виде \texttt{S-выражений}. 
Поэтому программы могут обрабатывать и преобразовывать другие программы и даже самих себя. 
В процессе трансляции можно введенное и сформированное в результате вычислений выражение данных проинтерпретировать в качестве программы и непосредственно выполнить. 
Так как программа представляет собой S-выражение, в памяти она представлена либо как атом (5 указателей; форма представления атома в памяти), либо списковой ячейкой (бинарный узел; 2 указателя).

\subsection*{2. Трактовка элементов списка}

Первый аргумент списка, который поступает на вход интерпретатору, трактуется как имя функции, остальные -- как аргументы этой функции.

\subsection*{3. Порядок реализации программы}

Программа в языке \texttt{LISP} представляется \texttt{S-выражением}, которое передается интерпретатору -- функции \texttt{eval}, которая выводит последний, полученный после обработки S-выражения, результат.
Работа функции \texttt{eval} представлена на картинке ниже.

\subsection*{4. Способы определения функций}

С помощью макро определения \texttt{defun} или с использованием Лямбда-нотации (функция без имени).
	
\bibliographystyle{utf8gost705u}  % стилевой файл для оформления по ГОСТу
	
\bibliography{51-biblio}          % имя библиографической базы (bib-файла)
	
\end{document}