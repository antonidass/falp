\documentclass[12pt]{report}
\usepackage[utf8]{inputenc}
\usepackage[russian]{babel}
%\usepackage[14pt]{extsizes}
\usepackage{listings}
\usepackage{graphicx}
\usepackage{amsmath,amsfonts,amssymb,amsthm,mathtools} 
\usepackage{pgfplots}
\usepackage{filecontents}
\usepackage{float}
\usepackage{indentfirst}
\usepackage{eucal}
\usepackage{enumitem}
\usepackage{tasks}

%s\documentclass[openany]{book}
\frenchspacing

\usepackage{indentfirst} % Красная строка

\usetikzlibrary{datavisualization}
\usetikzlibrary{datavisualization.formats.functions}

\usepackage{amsmath}


% Для листинга кода:
\lstset{ %
	language=c,                 % выбор языка для подсветки (здесь это С)
	basicstyle=\small\sffamily, % размер и начертание шрифта для подсветки кода
	numbers=left,               % где поставить нумерацию строк (слева\справа)
	numberstyle=\tiny,           % размер шрифта для номеров строк
	stepnumber=1,                   % размер шага между двумя номерами строк
	numbersep=5pt,                % как далеко отстоят номера строк от подсвечиваемого кода
	showspaces=false,            % показывать или нет пробелы специальными отступами
	showstringspaces=false,      % показывать или нет пробелы в строках
	showtabs=false,             % показывать или нет табуляцию в строках
	frame=single,              % рисовать рамку вокруг кода
	tabsize=2,                 % размер табуляции по умолчанию равен 2 пробелам
	captionpos=t,              % позиция заголовка вверху [t] или внизу [b] 
	breaklines=true,           % автоматически переносить строки (да\нет)
	breakatwhitespace=false, % переносить строки только если есть пробел
	escapeinside={\#*}{*)}   % если нужно добавить комментарии в коде
}


\usepackage[left=2cm,right=2cm, top=2cm,bottom=2cm,bindingoffset=0cm]{geometry}
% Для измененных титулов глав:
\usepackage{titlesec, blindtext, color} % подключаем нужные пакеты
\definecolor{gray75}{gray}{0.75} % определяем цвет
\newcommand{\hsp}{\hspace{20pt}} % длина линии в 20pt
% titleformat определяет стиль
\titleformat{\chapter}[hang]{\Huge\bfseries}{\thechapter\hsp\textcolor{gray75}{|}\hsp}{0pt}{\Huge\bfseries}


% plot
\usepackage{pgfplots}
\usepackage{filecontents}
\usetikzlibrary{datavisualization}
\usetikzlibrary{datavisualization.formats.functions}

\begin{document}
	%\def\chaptername{} % убирает "Глава"
	\thispagestyle{empty}
	\begin{titlepage}
		\noindent \begin{minipage}{0.15\textwidth}
			\includegraphics[width=\linewidth]{img/b_logo}
		\end{minipage}
		\noindent\begin{minipage}{0.9\textwidth}\centering
			\textbf{Министерство науки и высшего образования Российской Федерации}\\
			\textbf{Федеральное государственное бюджетное образовательное учреждение высшего образования}\\
			\textbf{~~~«Московский государственный технический университет имени Н.Э.~Баумана}\\
			\textbf{(национальный исследовательский университет)»}\\
			\textbf{(МГТУ им. Н.Э.~Баумана)}
		\end{minipage}
		
		\noindent\rule{18cm}{3pt}
		\newline\newline
		\noindent ФАКУЛЬТЕТ $\underline{\text{«Информатика и системы управления»}}$ \newline\newline
		\noindent КАФЕДРА $\underline{\text{«Программное обеспечение ЭВМ и информационные технологии»}}$\newline\newline\newline\newline\newline
		
		\begin{center}
			\noindent\begin{minipage}{1.1\textwidth}\centering
				\Large\textbf{  Отчет по лабораторной работе №7}\newline
				\textbf{по дисциплине <<Функциональное и логическое}\newline
				\textbf{~~~программирование>>}\newline\newline
			\end{minipage}
		\end{center}
		
		\noindent\textbf{Тема} $\underline{\text{Рекурсивные функции}}$\newline\newline
		\noindent\textbf{Студент} $\underline{\text{Криков А.В.~~~~~~~~~~~~~~~~~~~~~~~~~~~~~~~~~~~~~~~~~~}}$\newline\newline
		\noindent\textbf{Группа} $\underline{\text{ИУ7-63Б~~~~~~~~~~~~~~~~~~~~~~~~~~~~~~~~~~~~~~~~~~~~~~~~~~}}$\newline\newline
		\noindent\textbf{Оценка (баллы)} $\underline{\text{~~~~~~~~~~~~~~~~~~~~~~~~~~~~~~~~~~~~~~~~~~~~~~~~~}}$\newline\newline
		\noindent\textbf{Преподаватель} $\underline{\text{Толпинская Н.Б., Строганов Ю.В.~~~~~~~~~~~~~~~~~~~~~~~~~~~~}}$\newline\newline\newline
		
		\begin{center}
			\vfill
			Москва~---~\the\year
			~г.
		\end{center}
	\end{titlepage}
	
	


\chapter*{Практические задания}

	
\section*{Задание 1}
Написать хвостовую рекурсивную функцию \texttt{my-reverse}, которая развернет верхний уровень своего списка-аргумента lst.

\begin{lstlisting}[language=Lisp]
(defun my-reverse (lst &optional res)
	(if lst
		(my-reverse (cdr lst) (cons (car lst) res))
		res))
\end{lstlisting}


\section*{Задание 2}
Написать функцию, которая возвращает первый элемент списка -аргумента, который сам является непустым списком.


\begin{lstlisting}[language=Lisp]
(defun find-first-deep (lst)
	(cond ((and (listp (car lst)) lst) (caar lst))
	  (lst (find-first-deep (cdr lst)))
	  (T ())))
  
			  
\end{lstlisting}



\section*{Задание 3}
Написать функцию, которая выбирает из заданного списка только те числа, которые больше 1 и меньше 10. 

\begin{lstlisting}[language=Lisp]
(defun find-nums (lst n1 n2 &optional res)
	(if lst
		(find-nums (cdr lst) n1 n2 (if (< n1 (car lst) n2) (cons (car lst) res) res))
		(reverse res)))
\end{lstlisting}


\section*{Задание 4}
Напишите рекурсивную функцию, которая умножает на заданное число-аргумент все числа из заданного списка-аргумента, когда
\begin{enumerate}
	\item Все элементы списка --- числа;
	\item Элементы списка -- любые объекты.
\end{enumerate}


\begin{lstlisting}[language=Lisp]
(defun mult-nums (lst n &optional res)
	(if lst
		(mult-nums (cdr lst) n (cons (* (car lst) n) res))
		(reverse res)))
  
(defun mult-nums (lst n &optional res)
	(if lst
		(mult-nums (cdr lst) n (cons (cond ((listp (car lst)) (mult-nums (car lst) n))
					   ((numberp (car lst)) (* (car lst) n))
					   (T (car lst)))
					 res))
		(reverse res)))
\end{lstlisting}



\section*{Задание 5}
Напишите функцию, select-between, которая из списка-аргумента, содержащего только
числа, выбирает только те, которые расположены между двумя указанными границами аргументами и возвращает их в виде списка (упорядоченного по возрастанию списка чисел
(+ 2 балла)).



\begin{lstlisting}[language=Lisp]
(defun select-between (lst n1 n2 &optional res)
	(if lst
		(select-between (cdr lst) n1 n2 (if (< n1 (car lst) n2) (cons (car lst) res) res))
		res)) 

(defun select-between-sorted (lst n1 n2)
  	(sort (select-between lst n1 n2) #'<))

\end{lstlisting}


\section*{Задание 6}
Написать рекурсивную версию (с именем rec-add) вычисления суммы чисел заданного
списка:
\begin{enumerate}
	\item одноуровнего смешанного;
	\item структурированного.
\end{enumerate}


\begin{lstlisting}[language=Lisp]
(defun rec-add (lst &optional (res 0))
	(if lst
		(rec-add (cdr lst) (+ res (car lst)))
		res))
  
		
(defun rec-add (lst &optional (res 0))
	(if lst
		(rec-add (cdr lst) (cond ((listp (car lst)) (rec-add (car lst) res))
					 ((numberp (car lst)) (+ (car lst) res))
					 (T res)))
		res))

\end{lstlisting}






\section*{Задание 7}
Написать рекурсивную версию с именем \texttt{recnth} функции \texttt{nth}.


\begin{lstlisting}[language=Lisp]
(defun recnth (n lst)
	(if (= n 0) (car lst) (recnth (- n 1) (cdr lst))))
\end{lstlisting}



\section*{Задание 8}
Написать рекурсивную функцию \texttt{allodd}, которая возвращает t когда все элементы списка нечетные.

\begin{lstlisting}
(defun allodd (lst)
  	(if lst (if (oddp (car lst)) (allodd (cdr lst)) ()) T))
\end{lstlisting}



\section*{Задание 9}
Написать рекурсивную функцию, которая возвращает первое нечетное число из списка (структурированного), возможно создавая некоторые вспомогательные функции.


\begin{lstlisting}[language=Lisp]
(defun _first-odd (lst)
	(if lst
		(cond ((listp (car lst)) (or (_first-odd (car lst)) (_first-odd (cdr lst))))
		  ((numberp (car lst)) (if (oddp (car lst)) (car lst) (_first-odd (cdr lst)))))))
\end{lstlisting}

\section*{Задание 10}
Используя cons-дополняемую рекурсию с одним тестом завершения, написать функцию которая получает как аргумент список чисел, а возвращает список квадратов этих чисел в том же порядке.

\begin{lstlisting}[language=Lisp]
(defun sqr-lst (lst)
	(if lst (cons (* (car lst) (car lst)) (sqr-lst (cdr lst)))))
\end{lstlisting}
	

\bibliographystyle{utf8gost705u}  % стилевой файл для оформления по ГОСТу
	
\bibliography{51-biblio}          % имя библиографической базы (bib-файла)
	
\end{document}