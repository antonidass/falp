\documentclass[12pt]{report}
\usepackage[utf8]{inputenc}
\usepackage[russian]{babel}
%\usepackage[14pt]{extsizes}
\usepackage{listings}
\usepackage{graphicx}
\usepackage{amsmath,amsfonts,amssymb,amsthm,mathtools} 
\usepackage{pgfplots}
\usepackage{filecontents}
\usepackage{float}
\usepackage{indentfirst}
\usepackage{eucal}
\usepackage{enumitem}
\usepackage{tasks}

%s\documentclass[openany]{book}
\frenchspacing

\usepackage{indentfirst} % Красная строка

\usetikzlibrary{datavisualization}
\usetikzlibrary{datavisualization.formats.functions}

\usepackage{amsmath}


% Для листинга кода:
\lstset{ %
	language=c,                 % выбор языка для подсветки (здесь это С)
	basicstyle=\small\sffamily, % размер и начертание шрифта для подсветки кода
	numbers=left,               % где поставить нумерацию строк (слева\справа)
	numberstyle=\tiny,           % размер шрифта для номеров строк
	stepnumber=1,                   % размер шага между двумя номерами строк
	numbersep=5pt,                % как далеко отстоят номера строк от подсвечиваемого кода
	showspaces=false,            % показывать или нет пробелы специальными отступами
	showstringspaces=false,      % показывать или нет пробелы в строках
	showtabs=false,             % показывать или нет табуляцию в строках
	frame=single,              % рисовать рамку вокруг кода
	tabsize=2,                 % размер табуляции по умолчанию равен 2 пробелам
	captionpos=t,              % позиция заголовка вверху [t] или внизу [b] 
	breaklines=true,           % автоматически переносить строки (да\нет)
	breakatwhitespace=false, % переносить строки только если есть пробел
	escapeinside={\#*}{*)}   % если нужно добавить комментарии в коде
}


\usepackage[left=2cm,right=2cm, top=2cm,bottom=2cm,bindingoffset=0cm]{geometry}
% Для измененных титулов глав:
\usepackage{titlesec, blindtext, color} % подключаем нужные пакеты
\definecolor{gray75}{gray}{0.75} % определяем цвет
\newcommand{\hsp}{\hspace{20pt}} % длина линии в 20pt
% titleformat определяет стиль
\titleformat{\chapter}[hang]{\Huge\bfseries}{\thechapter\hsp\textcolor{gray75}{|}\hsp}{0pt}{\Huge\bfseries}


% plot
\usepackage{pgfplots}
\usepackage{filecontents}
\usetikzlibrary{datavisualization}
\usetikzlibrary{datavisualization.formats.functions}

\begin{document}
	%\def\chaptername{} % убирает "Глава"
	\thispagestyle{empty}
	\begin{titlepage}
		\noindent \begin{minipage}{0.15\textwidth}
			\includegraphics[width=\linewidth]{img/b_logo}
		\end{minipage}
		\noindent\begin{minipage}{0.9\textwidth}\centering
			\textbf{Министерство науки и высшего образования Российской Федерации}\\
			\textbf{Федеральное государственное бюджетное образовательное учреждение высшего образования}\\
			\textbf{~~~«Московский государственный технический университет имени Н.Э.~Баумана}\\
			\textbf{(национальный исследовательский университет)»}\\
			\textbf{(МГТУ им. Н.Э.~Баумана)}
		\end{minipage}
		
		\noindent\rule{18cm}{3pt}
		\newline\newline
		\noindent ФАКУЛЬТЕТ $\underline{\text{«Информатика и системы управления»}}$ \newline\newline
		\noindent КАФЕДРА $\underline{\text{«Программное обеспечение ЭВМ и информационные технологии»}}$\newline\newline\newline\newline\newline
		
		\begin{center}
			\noindent\begin{minipage}{1.1\textwidth}\centering
				\Large\textbf{  Отчет по лабораторной работе №3}\newline
				\textbf{по дисциплине <<Функциональное и логическое}\newline
				\textbf{~~~программирование>>}\newline\newline
			\end{minipage}
		\end{center}
		
		\noindent\textbf{Тема} $\underline{\text{Работа интерпретатора Lisp}}$\newline\newline
		\noindent\textbf{Студент} $\underline{\text{Криков А.В.~~~~~~~~~~~~~~~~~~~~~~~~~~~~~~~~~~~~~~~~~~}}$\newline\newline
		\noindent\textbf{Группа} $\underline{\text{ИУ7-63Б~~~~~~~~~~~~~~~~~~~~~~~~~~~~~~~~~~~~~~~~~~~~~~~~~~}}$\newline\newline
		\noindent\textbf{Оценка (баллы)} $\underline{\text{~~~~~~~~~~~~~~~~~~~~~~~~~~~~~~~~~~~~~~~~~~~~~~~~~}}$\newline\newline
		\noindent\textbf{Преподаватель} $\underline{\text{Толпинская Н.Б., Строганов Ю.В.~~~~~~~~~~~~~~~~~~~~~~~~~~~~}}$\newline\newline\newline
		
		\begin{center}
			\vfill
			Москва~---~\the\year
			~г.
		\end{center}
	\end{titlepage}
	
	


\chapter*{Практические задания}

\section*{Задание 1}
\subsection*{Постановка задачи}
	
Написать функцию, которая принимает целое число и возвращает первое четное число, не меньшее аргумента. 

\subsection*{Решение}

\begin{lstlisting}[label=first,caption=Решение задания №1, language=lisp]
(defun f1 (x) (if (= (mod x 2) 1) (+ x 1) x))
\end{lstlisting}

\section*{Задание №2}
\subsection*{Постановка задачи}
Написать функцию, которая принимает число и возвращает число того же знака, но с модулем на 1 больше модуля аргумента.

\subsection*{Решение}

\begin{lstlisting}[label=second,caption=Решение задания №2, language=lisp]
(defun abs-plus-one (x)
	(if (> x 0) (+ x 1) (- x 1)))
\end{lstlisting}

\section*{Задание №3}
\subsection*{Постановка задачи}
Написать функцию, которая принимает два числа и возвращает список из этих чисел, расположенный по возрастанию.

\subsection*{Решение}
\begin{lstlisting}[label=third,caption=Решение задания №3, language=lisp]
(defun sorted-pair-list (fst snd)
	(if (> fst snd) (list fst snd) (list snd fst)))
\end{lstlisting}

\section*{Задание №4}
\subsection*{Постановка задачи}
Написать функцию, которая принимает три числа и возвращает Т только тогда, когда первое число расположено между вторым и третьим.

\subsection*{Решение}
\begin{lstlisting}[label=4,caption=Решение задания №4, language=lisp]
	(defun between (a b c) (or (and (> a b) (< a c)) (and (> a c) (< a b))))
\end{lstlisting}

\section*{Задание №5}
\subsection*{Постановка задачи}
Каков результат вычисления следующих выражений?

\subsection*{Решение}
\begin{lstlisting}[label=5,caption=Решение задания №5, language=lisp]
(and `fee `fie `foe) -> FOE
(or nil `fie `foe) -> FIE
(and (equal `abc `abc) `yes) -> YES
(or `fee `fie `foe) -> FEE
(and nil `fie `foe) -> NIL
(or (equal `abc `abc) `yes) -> T
\end{lstlisting}

\section*{Задание №6}
\subsection*{Постановка задачи}
Написать предикат, который принимает два числа аргумента и возвращает Т, если первое число не меньше второго.

\subsection*{Решение}
\begin{lstlisting}[label=6,caption=Решение задания №6, language=lisp]
(defun ge (a b)
	(if (>= a b) T Nil))
\end{lstlisting}

\section*{Задание №7}
\subsection*{Постановка задачи}
Какой из следующих двух вариантов предиката ошибочен и почему? 

\subsection*{Решение}
\begin{lstlisting}[label=7,caption=Решение задания №7, language=lisp]
(defun pred1 (x)
	(and (numberp x) (plusp x))); OK
	
(defun pred2 (x)
	(and (plusp x) (numberp x))); RUNTIME ERROR
\end{lstlisting}

Второй вариант ошибочен, т.к. если в функцию будет передано не число и на него будет применена функция \textbf{plusp} (которая работает только с числовыми значениями), интерпретатор выдаст ошибку.

\section*{Задание №8}
\subsection*{Постановка задачи}
Решить задачу 4, используя для ее решения конструкции IF, COND, AND/OR.

\subsection*{Решение}
\begin{lstlisting}[label=8,caption=Решение задания №8, language=lisp]
(defun is-between-if (f s th)
(if (> f s)
	(if (< f th)
	t
	(if (< f s)
		(if (> f th)
		t
		nil)
		nil))
	(if (> f th)
	t)))
	
(defun is-between-cond(f s th)
	(cond ((> f s) (cond ((< f th))
						 (t nil)))
		    ((> f th) t)))

(defun beetween (a b c) (or (and (> a b) (< a c)) (and (> a c) (< a b))))

\end{lstlisting}


\section*{Задание №9}
\subsection*{Постановка задачи}
Переписать функцию how-alike, приведенную в лекции и использующую COND, используя конструкции IF, AND/OR.

\subsection*{Решение}
\begin{lstlisting}[label=9,caption=Решение задания №9, language=lisp]
(defun how-alike-cond (x y)
	(cond ((or (= x y) (equal x y)) `the_same)
	((and (oddp x) (oddp y)) `both_odd)
	((and (evenp x) (evenp y)) `both_even)
	(T `difference)))

(defun how-alike-if (x y)
	(if (if (= x y) (equal x y)) `the_same (
		if (if (oddp x) (oddp y)) `both_odd (
			if (if (evenp x) (evenp y)) `both_even `difference))))

(defun how-alike-andor (x y)
	(or (and (or (= x y) (equal x y)) `the_same)
	(and (oddp x) (oddp y) `both_odd)
	(and (evenp x) (evenp y) `both_even)
	`difference))
\end{lstlisting}


	


\chapter*{Контрольные вопросы}

\begin{enumerate}[wide=0pt]
\item \textit{Базис Lisp.} \\
\begin{enumerate}
	\item атомы и структуры (представляющиеся бинарными узлами);
	\item несколько базовых функций и функционалов: встроенные --- примитивные функции (atom, eq, cons, car, cdr); специальные функции и функционалы (quote, cond, lambda, eval, apply, funcall).
\end{enumerate}
\item \textit{Классификация функций.} \\
\begin{enumerate}
	\item чистые (математические) функции: имеют фиксированное количество
	аргументов и в качестве возврата единственное значение;
	\item рекурсивные функции;
	\item специальные функции (формы): имеют произвольное количество
	аргументов, либо эти аргументы обрабатываются не все одинаково;
	\item псевдофункции: функции, эффект которых виден на внешних
	устройствах;
	\item функции с вариантными значениями, из которых выбирается одно;
	\item функции высших порядков (функционалы) используются для
	построения синтаксически-управляемых программ, в качестве одного
	из аргументов принимают описание функции.
\end{enumerate}
\item \textit{Способы создания функций.} \\
Обычно функции определяются при помощи макроса DEFUN. В качестве имени может использоваться любой символ. Как правило, имена функций содержат только буквы, цифры и знак минус. Список параметров функции определяет переменные, которые будут использоваться для хранения аргументов, переданных при вызове функции. Тело DEFUN состоит из любого числа выражений Lisp. 
\item \textit{Работа функций and, or, if, cond.} \\
Сигнатура функции \textbf{cond}:

\indent(cond (предикат-1 результат-1)) \newline
\indent(предикат-2 результат-2) \newline
\indent...\newline
\indent(предикат-n результат-n)\newline

\indent Работа функции \textbf{cond}: 

сначала просматриваются все предикаты в порядке следования, и если хоть один из них истинный, то cond возвращает результат, связанный с этим предикатом. Если ни один предикат не был истинным, то она вернет Nil. 

Сигнатура функции \textbf{if}: 

(if условие выражение-1 выражение-2)\newline

\indent Работа функции \textbf{if}: 

если условие истинно (T), то выполняется выражение-1, иначе (Nil) – выражение-2\newline

Сигнатура функции \textbf{and}: 

(and выражение-1 выражение-2 ... выражение-n)\newline

\indent Работа функции \textbf{and}: 

результат функции будет истинным, если все ее выражения истинны. В таком случае в качестве результата вернется значение выражения-n. В случае, если хотя бы одно выражение ложно (Nil), вычисление последующих выражений не производится и результатом функции является Nil.\newline

Сигнатура функции \textbf{or}: 

(or выражение-1 выражение-2 ... выражение-n)\newline

Работа функции \textbf{or}: 

результат функции будет ложным (Nil), если все ее выражения ложны. В случае, если хотя бы одно выражение истинно, вычисление последующих выражений не производится и результатом функции является значения выражения, которое первым в списке аргументов дало в результате истину.\newline
\end{enumerate}
	
\bibliographystyle{utf8gost705u}  % стилевой файл для оформления по ГОСТу
	
\bibliography{51-biblio}          % имя библиографической базы (bib-файла)
	
\end{document}