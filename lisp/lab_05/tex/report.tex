\documentclass[12pt]{report}
\usepackage[utf8]{inputenc}
\usepackage[russian]{babel}
%\usepackage[14pt]{extsizes}
\usepackage{listings}
\usepackage{graphicx}
\usepackage{amsmath,amsfonts,amssymb,amsthm,mathtools} 
\usepackage{pgfplots}
\usepackage{filecontents}
\usepackage{float}
\usepackage{indentfirst}
\usepackage{eucal}
\usepackage{enumitem}
\usepackage{tasks}

%s\documentclass[openany]{book}
\frenchspacing

\usepackage{indentfirst} % Красная строка

\usetikzlibrary{datavisualization}
\usetikzlibrary{datavisualization.formats.functions}

\usepackage{amsmath}


% Для листинга кода:
\lstset{ %
	language=c,                 % выбор языка для подсветки (здесь это С)
	basicstyle=\small\sffamily, % размер и начертание шрифта для подсветки кода
	numbers=left,               % где поставить нумерацию строк (слева\справа)
	numberstyle=\tiny,           % размер шрифта для номеров строк
	stepnumber=1,                   % размер шага между двумя номерами строк
	numbersep=5pt,                % как далеко отстоят номера строк от подсвечиваемого кода
	showspaces=false,            % показывать или нет пробелы специальными отступами
	showstringspaces=false,      % показывать или нет пробелы в строках
	showtabs=false,             % показывать или нет табуляцию в строках
	frame=single,              % рисовать рамку вокруг кода
	tabsize=2,                 % размер табуляции по умолчанию равен 2 пробелам
	captionpos=t,              % позиция заголовка вверху [t] или внизу [b] 
	breaklines=true,           % автоматически переносить строки (да\нет)
	breakatwhitespace=false, % переносить строки только если есть пробел
	escapeinside={\#*}{*)}   % если нужно добавить комментарии в коде
}


\usepackage[left=2cm,right=2cm, top=2cm,bottom=2cm,bindingoffset=0cm]{geometry}
% Для измененных титулов глав:
\usepackage{titlesec, blindtext, color} % подключаем нужные пакеты
\definecolor{gray75}{gray}{0.75} % определяем цвет
\newcommand{\hsp}{\hspace{20pt}} % длина линии в 20pt
% titleformat определяет стиль
\titleformat{\chapter}[hang]{\Huge\bfseries}{\thechapter\hsp\textcolor{gray75}{|}\hsp}{0pt}{\Huge\bfseries}


% plot
\usepackage{pgfplots}
\usepackage{filecontents}
\usetikzlibrary{datavisualization}
\usetikzlibrary{datavisualization.formats.functions}

\begin{document}
	%\def\chaptername{} % убирает "Глава"
	\thispagestyle{empty}
	\begin{titlepage}
		\noindent \begin{minipage}{0.15\textwidth}
			\includegraphics[width=\linewidth]{img/b_logo}
		\end{minipage}
		\noindent\begin{minipage}{0.9\textwidth}\centering
			\textbf{Министерство науки и высшего образования Российской Федерации}\\
			\textbf{Федеральное государственное бюджетное образовательное учреждение высшего образования}\\
			\textbf{~~~«Московский государственный технический университет имени Н.Э.~Баумана}\\
			\textbf{(национальный исследовательский университет)»}\\
			\textbf{(МГТУ им. Н.Э.~Баумана)}
		\end{minipage}
		
		\noindent\rule{18cm}{3pt}
		\newline\newline
		\noindent ФАКУЛЬТЕТ $\underline{\text{«Информатика и системы управления»}}$ \newline\newline
		\noindent КАФЕДРА $\underline{\text{«Программное обеспечение ЭВМ и информационные технологии»}}$\newline\newline\newline\newline\newline
		
		\begin{center}
			\noindent\begin{minipage}{1.1\textwidth}\centering
				\Large\textbf{  Отчет по лабораторной работе №5}\newline
				\textbf{по дисциплине <<Функциональное и логическое}\newline
				\textbf{~~~программирование>>}\newline\newline
			\end{minipage}
		\end{center}
		
		\noindent\textbf{Тема} $\underline{\text{Использование управляющих структур, работа со списками}}$\newline\newline
		\noindent\textbf{Студент} $\underline{\text{Криков А.В.~~~~~~~~~~~~~~~~~~~~~~~~~~~~~~~~~~~~~~~~~~}}$\newline\newline
		\noindent\textbf{Группа} $\underline{\text{ИУ7-63Б~~~~~~~~~~~~~~~~~~~~~~~~~~~~~~~~~~~~~~~~~~~~~~~~~~}}$\newline\newline
		\noindent\textbf{Оценка (баллы)} $\underline{\text{~~~~~~~~~~~~~~~~~~~~~~~~~~~~~~~~~~~~~~~~~~~~~~~~~}}$\newline\newline
		\noindent\textbf{Преподаватель} $\underline{\text{Толпинская Н.Б., Строганов Ю.В.~~~~~~~~~~~~~~~~~~~~~~~~~~~~}}$\newline\newline\newline
		
		\begin{center}
			\vfill
			Москва~---~\the\year
			~г.
		\end{center}
	\end{titlepage}
	
	


\chapter*{Практические задания}

	
\section*{Задание 1}
Написать функцию, которая по своему списку-аргументу \texttt{lst} определяет, является ли он палиндромом (то есть равны ли \texttt{lst} и \texttt{(reverse lst)}).

\begin{lstlisting}[language=Lisp]
	(defun pol(lst) 
	(equal lst (reverse lst)))
\end{lstlisting}


\section*{Задание 2}
Написать предикат \texttt{set-equal}, который возвращает \texttt{t}, если два его множества-аргумента содержат одни и те же элементы, порядок которых не имеет значения.

\begin{lstlisting}[language=Lisp]
(defun set-equal (set1 set2)
	(and (= (length set1) (length set2))
		 (every #'(lambda (elem) (member elem set2 :test #'equal)) 
		  set1)
		 (every #'(lambda (elem) (member elem set1 :test #'equal)) 
		  set2)))
\end{lstlisting}


\section*{Задание 3}
Напишите свои необходимые функции, которые обрабатывают таблицу из 4-х точечных пар: (страна . столица), и возвращают по стране --- столицу, а по столице --- страну .
    \begin{lstlisting}[language=Lisp]
(defun search-country (captl table)
	(cond ((null table) nil)
		((eq captl (cdar table)) (caar table))
		(t (search-country captl (cdr table)))))

(defun search-capital (cntry table)
	(cond ((null table) nil)
			((eq cntry (caar table)) (cdar table))
			(t (search-capital cntry (cdr table)))))
			
(defun search-country-rassoc (captl table)
	(car (rassoc captl table)))

(defun search-capital-assoc (cntry table)
	(cdr (assoc cntry table)))
    \end{lstlisting}


\section*{Задание 4}
Напишите функцию \texttt{swap-first-last}, которая переставляет в списке-аргументе первый и последний элементы.
\begin{lstlisting}[language=Lisp]
(defun swap-first-last (lst)
	(let* ((tmp (reverse (cdr lst)))
			(mid (reverse (cdr tmp))))
			(cons (car tmp) (append mid (list (car lst))))))
\end{lstlisting}


\section*{Задание 5}
Напишите функцию \texttt{swap-two-elements}, которая переставляет в списке-аргументе два указанных своими порядковыми номерами элемента в этом списке.

\begin{lstlisting}[language=Lisp]
(defun swap-two-elements(lst i j) 
	(if (or (> i (length lst)) (> j (length lst)))
		lst
		(let ((fst (nth i lst))
			(snd (nth j lst)))
			(progn (setf (nth i lst) snd) (setf (nth j lst) fst) lst))))
\end{lstlisting}


\section*{Задание 6}
Напишите две функции, \texttt{swap-to-left} и \texttt{swap-to-right}, которые производят одну круговую перестановку в списке-аргументе влево и вправо, соответственно.
    
\begin{lstlisting}[language=Lisp]
(defun swap-to-left(lst)
	(let* ((tmp (cdr lst))
			(fnl (car lst)))
			(append tmp (list fnl))))
		
(defun swap-to-right(lst)
	(let* ((tmp (reverse (cdr (reverse lst))))
			(fnl (car (reverse lst))))
			(cons fnl tmp)))
\end{lstlisting}






\section*{Задание 7}
Напишите функцию, которая добавляет к множеству двухэлементных списков новый двухэлементный список, если его там нет.

\begin{lstlisting}[language=Lisp]
(defun add-list(src dest)
	(if (some #'(lambda (pair) (equal dest pair)) src)
		src
		(append src (list dest))))		
\end{lstlisting}



\section*{Задание 8}
Напишите функцию, которая умножает на заданное число-аргумент первый числовой элемент списка из заданного 3-х элементного списка-аргумента, когда:		
\begin{enumerate}
	\item \textit{все элементы списка --- числа,}
	\item \textit{элементы списка -- любые объекты.}
\end{enumerate}


\begin{lstlisting}[language=Lisp]    
;; (a)
(defun multiply-lst-num (lst mul)
		(mapcar #'(lambda (elem) (* elem mul)) lst))
	
;; (b)
(defun multiply-lst (lst mul)
		(mapcar #'(lambda (elem) (cond ((listp elem) (multiply-lst elem mul))
										((numberp elem) (* elem mul))
										(t elem)))
		lst))

\end{lstlisting}




\section*{Задание 9}
Напишите функцию, select-between, которая из списка-аргумента из 5 чисел выбирает только те, которые расположены между двумя указанными границами-аргументами и возвращает их в виде списка (упорядоченного по возрастанию списка чисел (+ 2 балла)).

\begin{lstlisting}[language=Lisp]
(defun select-between (lst left right)
		(sort (reduce #'(lambda (res el) 
				(if (and (> el left) (< el right))
				(cons el res)
				res)) lst :initial-value ()) #'<))
\end{lstlisting}


\section*{Контрольные вопросы}

\subsection*{1. Cтруктуроразрушающие и не разрушающие структуру списка функции.}

Не разрушающие структуру функции не меняют сам объект-аргумент, а создают его копию. Например, reverse, append.
Разрушающие структуру функции меняют объект-аргумент, и получить исходный уже невозможно. Такие функции начинаются с префикса n: nconc, nreverse. 


\subsection*{2. Отличие в работе функций cons, list, append, nconc и в их результате.}

\begin{enumerate}
	\item cons конструирует точечную пару или список, в зависимости от второго аргумента. Является чистой функцией, принимает два аргумента;
	\item list является формой, принимает произвольное количество аргументов и составляет из них список. Результатом работы всегда является список.
	\item append является формой, принимает произвольное количество аргументов, создает копию для всех, кроме последнего, при этом последний элемент каждого списка-аргумента ссылается на первый элемент следующего по порядку списка-аргумента (или его копию, в зависимости от расположения).
	\item nconc возвращает список с элементами из всех списков-аргументов (по порядку). Принцип работы: устанавливает cdr последней ячейки каждого списка в начало следующего списка. Последний аргумент может быть объектом любого типа. Вызванная без аргументов, возвращает nil.
\end{enumerate} 


\bibliographystyle{utf8gost705u}  % стилевой файл для оформления по ГОСТу
	
\bibliography{51-biblio}          % имя библиографической базы (bib-файла)
	
\end{document}